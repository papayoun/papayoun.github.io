\documentclass[]{article}
\usepackage{lmodern}
\usepackage{amssymb,amsmath}
\usepackage{ifxetex,ifluatex}
\usepackage{fixltx2e} % provides \textsubscript
\ifnum 0\ifxetex 1\fi\ifluatex 1\fi=0 % if pdftex
  \usepackage[T1]{fontenc}
  \usepackage[utf8]{inputenc}
\else % if luatex or xelatex
  \ifxetex
    \usepackage{mathspec}
  \else
    \usepackage{fontspec}
  \fi
  \defaultfontfeatures{Ligatures=TeX,Scale=MatchLowercase}
\fi
% use upquote if available, for straight quotes in verbatim environments
\IfFileExists{upquote.sty}{\usepackage{upquote}}{}
% use microtype if available
\IfFileExists{microtype.sty}{%
\usepackage{microtype}
\UseMicrotypeSet[protrusion]{basicmath} % disable protrusion for tt fonts
}{}
\usepackage[margin=1in]{geometry}
\usepackage{hyperref}
\hypersetup{unicode=true,
            pdftitle={Echantillonnage préférentiel},
            pdfauthor={Pierre Gloaguen},
            pdfborder={0 0 0},
            breaklinks=true}
\urlstyle{same}  % don't use monospace font for urls
\usepackage{graphicx,grffile}
\makeatletter
\def\maxwidth{\ifdim\Gin@nat@width>\linewidth\linewidth\else\Gin@nat@width\fi}
\def\maxheight{\ifdim\Gin@nat@height>\textheight\textheight\else\Gin@nat@height\fi}
\makeatother
% Scale images if necessary, so that they will not overflow the page
% margins by default, and it is still possible to overwrite the defaults
% using explicit options in \includegraphics[width, height, ...]{}
\setkeys{Gin}{width=\maxwidth,height=\maxheight,keepaspectratio}
\IfFileExists{parskip.sty}{%
\usepackage{parskip}
}{% else
\setlength{\parindent}{0pt}
\setlength{\parskip}{6pt plus 2pt minus 1pt}
}
\setlength{\emergencystretch}{3em}  % prevent overfull lines
\providecommand{\tightlist}{%
  \setlength{\itemsep}{0pt}\setlength{\parskip}{0pt}}
\setcounter{secnumdepth}{5}
% Redefines (sub)paragraphs to behave more like sections
\ifx\paragraph\undefined\else
\let\oldparagraph\paragraph
\renewcommand{\paragraph}[1]{\oldparagraph{#1}\mbox{}}
\fi
\ifx\subparagraph\undefined\else
\let\oldsubparagraph\subparagraph
\renewcommand{\subparagraph}[1]{\oldsubparagraph{#1}\mbox{}}
\fi

%%% Use protect on footnotes to avoid problems with footnotes in titles
\let\rmarkdownfootnote\footnote%
\def\footnote{\protect\rmarkdownfootnote}

%%% Change title format to be more compact
\usepackage{titling}

% Create subtitle command for use in maketitle
\providecommand{\subtitle}[1]{
  \posttitle{
    \begin{center}\large#1\end{center}
    }
}

\setlength{\droptitle}{-2em}

  \title{Echantillonnage préférentiel}
    \pretitle{\vspace{\droptitle}\centering\huge}
  \posttitle{\par}
  \subtitle{Travaux dirigés}
  \author{Pierre Gloaguen}
    \preauthor{\centering\large\emph}
  \postauthor{\par}
    \date{}
    \predate{}\postdate{}
  

\begin{document}
\maketitle

{
\setcounter{tocdepth}{3}
\tableofcontents
}
\hypertarget{uxe9vuxe8nement-rare}{%
\section{Évènement rare}\label{uxe9vuxe8nement-rare}}

On veut estimer la probabilité \(p^*\) qu'une loi normale centrée
réduite dépasse la valeur 3.

\begin{enumerate}
\def\labelenumi{\arabic{enumi}.}
\item
  Pour un effort de Monte Carlo de taille \(M\), proposer un estimateur
  de Monte Carlo pour \(p^*\) se basant sur la loi \(\mathcal{N}(0, 1)\)
  .
\item
  Implémenter cet estimateur sur \texttt{R} pour un effort de Monte
  Carlo de taille \(M= 10000\)
\item
  On se propose d'utiliser un échantillonnage préférentiel pour estimer
  cette probabilité. On utilisera comme loi d'échantillonnage une loi
  exponentielle translatée de 3, de paramètre \(\lambda\) notée
  \(Y\sim t\mathcal{E}(3, \lambda)\), i.e., la variable aléatoire \(Y\)
  telle que \(Y - 3 \sim \mathcal{E}(\lambda)\) Calculer les poids
  d'importance associés, et proposer un estimateur de \(p^*\)
\item
  Ecrire la variance de l'estimateur comme fonction de \(\lambda\).
  Comment doit on choisir \(\lambda\)?
\item
  Donner un estimateur empirique de cette variance, et donner
  l'expression de l'intervalle de confiance à 95\% pour l'estimateur de
  \(p^*\).
\item
  Implémenter cet estimateur sur \texttt{R} avec \(\lambda = 3.5\) et le
  comparer à celui de Monte Carlo.
\end{enumerate}

\hypertarget{cas-probluxe9matique}{%
\section{Cas problématique}\label{cas-probluxe9matique}}

\hypertarget{premier-cas-jouet}{%
\subsection{Premier cas jouet}\label{premier-cas-jouet}}

Afin de montrer qu'un mauvais choix de loi d'échantillonnage peut
augmenter drastiquement la variance de l'estimateur, on peut prendre un
exemple très simple.

Supposons qu'on veuille estimer par méthode de Monte Carlo
\(\mathbb{E}[X]\) où \(X \sim \mathcal{N}(0, 1)\). On se propose de le
faire par méthode de Monte Carlo standard et par échantillonnage
préférentiel en tirant dans une loi \(Y \sim\mathcal{N}(\mu, 1)\) où
\(\mu\) est choisi par l'utilisateur.

\begin{enumerate}
\def\labelenumi{\arabic{enumi}.}
\item
  Quelle est la variance de l'estimateur de Monte Carlo standard?
\item
  Quelle est la variance de l'estimateur par échantillonnage
  préférentiel?
\end{enumerate}

\hypertarget{encore-pire}{%
\subsection{Encore pire!}\label{encore-pire}}

Soit \(X\) une variable aléatoire de densité
\(f_X(x) = 3x^{-4}\mathbf{1}_{x\geq 1}\) (on dit que \(X\) suit une loi
de Pareto de paramètres (1, 3)).

\begin{enumerate}
\def\labelenumi{\arabic{enumi}.}
\item
  Calculer \(p = \mathbb{P}(X>10)\)
\item
  Supposons qu'on veuille estimer \(p\) par méthode de Monte Carlo.
  Proposer une méthode de simulation de \(X\) par méthode d'inversion.
  En déduire un estimateur de Monte Carlo standard.
\item
  Représenter graphiquement les performances empiriques de cet
  estimateur pour un effot Monte Carlo de \(M= 10^4\).
\item
  On propose maintenant un estimateur par échantillonnage préférentiel
  comme dans l'exercice précédent. On choisit comme densité
  d'échantillonnage celle d'une loi exponentielle translatée de 10, de
  paramètre \(\lambda = 1\). Donnez l'estimateur associé et mettez le en
  place sous R. Que constatez vous? Comment l'expliquez vous.
\item
  Que pouvez vous dire de la variance de cet estimateur?
\end{enumerate}

\hypertarget{marche-aluxe9atoire-du-joueur-optimiste}{%
\section{Marche aléatoire du joueur
optimiste}\label{marche-aluxe9atoire-du-joueur-optimiste}}

Le problème suivant peut être vu par le prisme d'un joueur dans un jeu à
espérance négative, qui décide \emph{a priori} de s'arrêter, soit quand
il est ruiné, soit quand ses gains ont dépassé le jackpot.

Soit \(X_0 = 0\) et \((X_n)_{n\geq 1}\) une suite de variables
aléatoires i.i.d. de loi \(\mathcal{N}(-\mu, 1)\) où \(\mu\) est une
constante strictement positive (gain pour une partie). On note
\(S_n = \sum_{0}^n X_n\) (somme des gains jusqu'à l'instant \(n\)). On
se donne un réel \(r < 0\) (représentant la ruine) et un réel \(j > 0\)
(représentant le jackpot). On considère le temps d'arrêt
\(T = \min\left\lbrace n, S_n \leq r \text{ ou } S_n \geq j\right\rbrace\).
On souhaite estimer la probabilité de sortir vainqueur, soit
\(p^* = \mathbb{P}(S_T \geq j)\).

\begin{enumerate}
\def\labelenumi{\arabic{enumi}.}
\item
  Proposer un algortihme de simulation de pour une trajectoire
  \((S_n)_{1\leq n \leq T}\).
\item
  Pour un effort de Monte Carlo de taille \(M\), proposer un estimateur
  pour \(p^*\).
\item
  Pour \(\mu =1, r = -50, j = 5\), implémenter cet estimateur, quelle
  est la valeur de \(\hat{p}^*\) obtenue pour \(M = 1000\)? Représentez
  une trajectoire d'une estimation (i.e., la valeur de l'estimation en
  fonction de \(M\))?
\item
  Pour une séquence observée \(s_1, \dots, s_T\) simulée grâce à la
  méthode de Monte Carlo ci dessus, donner la densité de l'échantillon,
  notée \(f(s_{1:T})\).
\item
  On considère maintenant la marche aléatoire où les \(X_n\) sont tirés
  selon la loi \(\mathcal{N}(\mu, 1)\) (n a toujours \(X_0 = 0\) et
  \(S_n = \sum_{0}^n X_n\)). Pour une séquence \(s_1, \dots, s_T\)
  simulée ainsi on note \(g(s_1, \dots, s_T)\), la densité de
  l'échantillon, écrire le rapport \(\frac{f(s_{1:T})}{g(s_{1:T})}\).
\item
  Proposer un estimateur par échantillonnage préférentiel pour estimer
  \(p^*\), basé sur la simulation selon la densité \(g\). L'estimateur
  sera noté \(\hat{p}_M^{IS}\). De cet estimateur, vous déduirez que
  \(p^* \leq \text{e}^{-2\mu j}\).
\item
  Implémenter cet algorithme, tracez la valeur de l'estimation ainsi que
  son intervalle de confiance en fonction de \(M\).
\end{enumerate}


\end{document}
