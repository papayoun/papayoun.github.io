\documentclass[]{article}
\usepackage{lmodern}
\usepackage{amssymb,amsmath}
\usepackage{ifxetex,ifluatex}
\usepackage{fixltx2e} % provides \textsubscript
\ifnum 0\ifxetex 1\fi\ifluatex 1\fi=0 % if pdftex
  \usepackage[T1]{fontenc}
  \usepackage[utf8]{inputenc}
\else % if luatex or xelatex
  \ifxetex
    \usepackage{mathspec}
  \else
    \usepackage{fontspec}
  \fi
  \defaultfontfeatures{Ligatures=TeX,Scale=MatchLowercase}
\fi
% use upquote if available, for straight quotes in verbatim environments
\IfFileExists{upquote.sty}{\usepackage{upquote}}{}
% use microtype if available
\IfFileExists{microtype.sty}{%
\usepackage{microtype}
\UseMicrotypeSet[protrusion]{basicmath} % disable protrusion for tt fonts
}{}
\usepackage[margin=1in]{geometry}
\usepackage{hyperref}
\hypersetup{unicode=true,
            pdftitle={Simulations de variables aléatoires},
            pdfauthor={Pierre Gloaguen},
            pdfborder={0 0 0},
            breaklinks=true}
\urlstyle{same}  % don't use monospace font for urls
\usepackage{graphicx,grffile}
\makeatletter
\def\maxwidth{\ifdim\Gin@nat@width>\linewidth\linewidth\else\Gin@nat@width\fi}
\def\maxheight{\ifdim\Gin@nat@height>\textheight\textheight\else\Gin@nat@height\fi}
\makeatother
% Scale images if necessary, so that they will not overflow the page
% margins by default, and it is still possible to overwrite the defaults
% using explicit options in \includegraphics[width, height, ...]{}
\setkeys{Gin}{width=\maxwidth,height=\maxheight,keepaspectratio}
\IfFileExists{parskip.sty}{%
\usepackage{parskip}
}{% else
\setlength{\parindent}{0pt}
\setlength{\parskip}{6pt plus 2pt minus 1pt}
}
\setlength{\emergencystretch}{3em}  % prevent overfull lines
\providecommand{\tightlist}{%
  \setlength{\itemsep}{0pt}\setlength{\parskip}{0pt}}
\setcounter{secnumdepth}{5}
% Redefines (sub)paragraphs to behave more like sections
\ifx\paragraph\undefined\else
\let\oldparagraph\paragraph
\renewcommand{\paragraph}[1]{\oldparagraph{#1}\mbox{}}
\fi
\ifx\subparagraph\undefined\else
\let\oldsubparagraph\subparagraph
\renewcommand{\subparagraph}[1]{\oldsubparagraph{#1}\mbox{}}
\fi

%%% Use protect on footnotes to avoid problems with footnotes in titles
\let\rmarkdownfootnote\footnote%
\def\footnote{\protect\rmarkdownfootnote}

%%% Change title format to be more compact
\usepackage{titling}

% Create subtitle command for use in maketitle
\providecommand{\subtitle}[1]{
  \posttitle{
    \begin{center}\large#1\end{center}
    }
}

\setlength{\droptitle}{-2em}

  \title{Simulations de variables aléatoires}
    \pretitle{\vspace{\droptitle}\centering\huge}
  \posttitle{\par}
  \subtitle{Travaux dirigés}
  \author{Pierre Gloaguen}
    \preauthor{\centering\large\emph}
  \postauthor{\par}
    \date{}
    \predate{}\postdate{}
  

\begin{document}
\maketitle

La plupart des exercices de cette feuille nécessite la confection de
programme en \texttt{R}.

Afin de garder trace de vos exercices, pensez à sauvegarder le script
associé, voire à répondre à l'exercice dans un fichier \emph{Rmarkdown}
(extension \texttt{.Rmd}).

L'environnement \texttt{Rstudio} est plus que vivement conseillé pour
programmer en \texttt{R}.

\hypertarget{guxe9nuxe9rateurs-pseudos-aluxe9atoires}{%
\section{Générateurs pseudos
aléatoires}\label{guxe9nuxe9rateurs-pseudos-aluxe9atoires}}

\hypertarget{guxe9nuxe9ration-de-loi-uniforme}{%
\subsection{Génération de loi
uniforme}\label{guxe9nuxe9ration-de-loi-uniforme}}

\begin{enumerate}
\def\labelenumi{\arabic{enumi}.}
\tightlist
\item
  À l'aide du logiciel R, programmez un générateur à congruences pour la
  loi uniforme. Ce générateur prendra la forme d'une fonction prenant en
  argument:
\end{enumerate}

\begin{itemize}
\tightlist
\item
  Un entier \texttt{n} donnant la taille de l'échantillon voulu.
\item
  4 entiers \texttt{a,\ m,\ c,\ x0} correspondant aux paramètres du
  générateurs vu en cours.
\end{itemize}

\begin{enumerate}
\def\labelenumi{\arabic{enumi}.}
\setcounter{enumi}{1}
\tightlist
\item
  À l'aide de cette fonction, générer un échantillon de taille 10000
  pour les valeurs
\end{enumerate}

\begin{itemize}
\tightlist
\item
  \texttt{a\ =\ 41358}
\item
  \texttt{m\ =\ 2\^{}31\ -1}
\item
  \texttt{c\ =\ 0}
\end{itemize}

et la graine de votre choix. Refaites la même procédure avec

\begin{itemize}
\tightlist
\item
  \texttt{a\ =\ 3}
\item
  \texttt{m\ =\ 2\^{}31\ -1}
\item
  \texttt{c\ =\ 0}
\end{itemize}

et

\begin{itemize}
\tightlist
\item
  \texttt{a\ =\ 101}
\item
  \texttt{m\ =\ 2311}
\item
  \texttt{c\ =\ 0}
\end{itemize}

Vous stockerez chacun des échantillons obtenus

\begin{enumerate}
\def\labelenumi{\arabic{enumi}.}
\setcounter{enumi}{2}
\item
  Pour chacun des échantillons obtenus, tracez l'histogramme empirique.
  Quels échantillons vous semblent tirés selon une loi uniforme
  \(U[0, 1]\)? En utilisant la fonction \texttt{ks.test}, effectuez un
  test de Kolmogorov-Smirnoff d'adéquation pour la loi uniforme. Que
  concluez vous sur la qualité des 3 générateurs?
\item
  Pour chacun des échantillons \((u_1,\dots, u_{10000})\) obtenus,
  tracez \(u_n\) en fonction de \(u_{n-1}\). Que pouvez vous conclure
  sur la qualité des 3 générateurs?
\end{enumerate}

\hypertarget{muxe9thode-dinversion}{%
\section{Méthode d'inversion}\label{muxe9thode-dinversion}}

\hypertarget{loi-exponentielle}{%
\subsection{Loi exponentielle}\label{loi-exponentielle}}

On rappelle qu'une variable aléatoire \(X\) est de loi exponentielle de
paramètre \(\lambda > 0\) si elle a pour fonction de densité
\(f_X(x) = \lambda\text{e}^{-\lambda x} \mathbf{1}_{x\geq0}\)

\begin{enumerate}
\def\labelenumi{\arabic{enumi}.}
\item
  En utilisant la méthode d'inversion, proposez un algorithme de
  simulation pour une variable aléatoire exponentielle.
\item
  Ecrire une fonction \texttt{R} mettant en oeuvre cette algorithme.
  Cette fonction prendra deux paramètres en entrée:
\end{enumerate}

\begin{itemize}
\tightlist
\item
  \texttt{n} La taille de l'échantillon;
\item
  \texttt{lambda} Le paramètre de la loi exponentielle
\end{itemize}

Vous testerez la qualité de votre fonction sur un échantillon de taille
10000, en comparant graphiquement l'histogramme empirique obtenu à la
densité de la loi exponentielle correspondante.

\hypertarget{loi-discruxe8te}{%
\subsection{Loi discrète}\label{loi-discruxe8te}}

\label{exo:inv:disc} On considère une variable aléatoire discrète \(X\)
à valeurs dans l'ensemble \(\left\lbrace 1,\dots, K\right\rbrace\), dont
la loi est définie par le vecteur de probabilité \((p_1, \dots, p_K)\),
i.e.: \begin{align}
\mathbb{P}(X = k) &= p_k\\
\sum_{k = 1}^K p_k &= 1
\end{align}

\begin{enumerate}
\def\labelenumi{\arabic{enumi}.}
\item
  Pour tout \(u \in ]0, 1[\), écrire l'expression de l'inverse
  généralisée de la fonction de répartition de \(X\).
\item
  En déduire un algorithme de simulation pour toute variable aléatoire
  discrète dans un ensemble fini.
\item
  Utilisez cet algorithme de simulation pour simuler un échantillon de
  taille 10000 loi binomiale de paramètres \(n = 10\) et \(p = 0.5\)
  avec \texttt{R}. Vous comparerez les fréquences obtenues avec les
  fréquences théoriques.
\end{enumerate}

\hypertarget{muxe9thode-de-rejet}{%
\section{Méthode de rejet}\label{muxe9thode-de-rejet}}

\hypertarget{simulation-dune-loi-de-poisson-pour-lambda-1}{%
\subsection{\texorpdfstring{Simulation d'une loi de Poisson pour
\(\lambda < 1\)}{Simulation d'une loi de Poisson pour \textbackslash{}lambda \textless{} 1}}\label{simulation-dune-loi-de-poisson-pour-lambda-1}}

\begin{enumerate}
\def\labelenumi{\arabic{enumi}.}
\item
  On utilisant le résultat de l'exercice \ref{exo:inv:disc}, proposez un
  algorithme pour simuler une variable aléatoire de Bernouilli de
  paramètre \(p \in ]0, 1[\).
\item
  En déduire un algorithme pour simuler une loi géométrique de paramètre
  \(p\) sur \(\mathbb{N}\).
\item
  On souhaite obtenir un échantillon d'une loi de de Poisson de
  paramètre \(\lambda \in ]0, 1[\) par méthode d'acceptation rejet. On
  se propose d'utiliser comme loi de proposition la loi géométrique sur
  \(\mathbb{N}\) de paramètre \(1 - \lambda\). Définir l'algorithme de
  rejet correspondant.
\item
  Quelle est la probabilité d'acceptation dans l'algorithme de rejet?
\item
  Faites une fonction \texttt{R} permettant de générer une loi
  géométrique de paramètre \(p\). Utiliser cette fonction dans une autre
  fonction \texttt{R} permettant de simuler selon une loi de Poisson de
  paramètre \(p \in ]0, 1[\). Simuler ainsi un échantillon de taille
  10000. Comparer la distribution obtenue à celle de la vraie loi.
\end{enumerate}

\hypertarget{loi-uniforme-sur-le-disque-unituxe9}{%
\subsection{Loi uniforme sur le disque
unité}\label{loi-uniforme-sur-le-disque-unituxe9}}

\begin{enumerate}
\def\labelenumi{\arabic{enumi}.}
\item
  À partir d'une variable aléatoire uniforme sur \([0, 1]\), proposez
  une transformation pour simuler une loi uniforme sur \([-1, 1]\).
\item
  Proposer une méthode d'acceptation rejet pour simuler, à partir de
  deux variables aléatoires indépendantes de loi uniforme sur
  \([-1, 1]\), une variable aléatoire uniforme sur le disque unité.
\item
  Quelle est la probabilité d'acceptation de l'algorithme?
\item
  Ecrire une fonction \texttt{R} mettant en place la génération de
  variables aléatoires sur le disque unité. Dans cet algorithme, gardez
  en mémoire le nombre d'essais nécessaire avant chaque acceptation.
\item
  Générer un échantillon de taille 10000. Vérifiez graphiquement que ces
  points sont uniformément répartis sur le disque unité. Vérifiez
  également que le nombre d'essais moyens avant acceptation est en
  adéquation avec ce qui est attendu.
\end{enumerate}

\hypertarget{proposition-optimale}{%
\subsection{Proposition optimale}\label{proposition-optimale}}

La loi normale tronquée de support \([b, +\infty[\) est définie par la
densité \(f\) proportionnelle, pour tout réel \(x\), à
\[f_1(x) = \exp\left\lbrace -\frac{(x-\mu)^2}{2\sigma^2}\right\rbrace\mathbf{1}_{x\geq b},
\quad\text{avec $\mu > 0$, $\sigma > 0$}.\]

On propose de simuler suivant la loi de densité \(f\) par une méthode de
rejet.

\hypertarget{muxe9thode-nauxefve}{%
\subsubsection{Méthode naïve}\label{muxe9thode-nauxefve}}

\begin{enumerate}
\def\labelenumi{\arabic{enumi}.}
\item
  On note \(\Phi\) la fonction de répartition de la loi normale centrée
  réduite. Montrer que \(f\) satisfait l'inégalité suivante pour tout
  réel \(x\):
  \[f(x)\leq \frac{1}{\sigma\sqrt{2\pi}\Phi(\frac{\mu-b}{\sigma})}\exp\left\lbrace -\frac{(x-\mu)^2}{2\sigma^2}\right\rbrace.\]
\item
  En déduire l'algorithme du rejet. Que peut-on dire du nombre d'essais
  moyen avant acceptation ?
\end{enumerate}

\hypertarget{une-distribution-instrumentale-alternative}{%
\subsubsection{Une distribution instrumentale
alternative}\label{une-distribution-instrumentale-alternative}}

On suppose que \(b > \mu\). On considère la loi exponentielle translatée
de \(b\), \(\tau\mathcal{E}(\lambda, b)\), de densité
\[g_{\lambda}(x) = \lambda e^{-\lambda(x-b)}\mathbf{1}_{x\geq b},\quad x\in\mathbb{R}.\]
3. Montrer pour \(x\geq b\) que
\[\frac{f_1(x)}{g_{\lambda}(x)}\leq \left\lbrace
\begin{array}{lr}
\frac{1}{\lambda}\exp\left\lbrace 
    \lambda(\mu-b)+\frac{(\lambda\sigma)^2}{2}
    \right\rbrace & \text{si } \mu+\lambda\sigma^2 > b,\\
    \frac{1}{\lambda}\exp\left\lbrace 
    -\frac{(b-\mu)^2}{2\sigma^2}
    \right\rbrace & \text{sinon}.
\end{array}
\right. \]

\begin{enumerate}
\def\labelenumi{\arabic{enumi}.}
\setcounter{enumi}{3}
\item
  Proposer une méthode de simulation de la loi de densité \(f\).
\item
  Calculer la valeur de \(\lambda^*\) telle que le temps moyen de calcul
  de la méthode proposée soit le plus petit possible.
\item
  En \texttt{R}, mettre en oeuvre les deux méthodes afin de constater
  empiriquement les différences.
\end{enumerate}

\hypertarget{muxe9thode-de-transformation}{%
\section{Méthode de transformation}\label{muxe9thode-de-transformation}}

\hypertarget{simulation-de-lois-gaussiennes.-algorithme-de-box-muller}{%
\subsection{Simulation de lois Gaussiennes. Algorithme de
Box-Muller}\label{simulation-de-lois-gaussiennes.-algorithme-de-box-muller}}

Soient \(X\) et \(Y\) deux variables aléatoires indépendantes de loi
\(\mathcal{N}(0, 1)\)

\begin{enumerate}
\def\labelenumi{\arabic{enumi}.}
\tightlist
\item
  Montrer que si \(U\) et \(V\) sont deux variables aléatoires
  indépendantes de loi \(\mathcal{U}[0, 1]\) alors le couple
  \[\left(\sqrt{- 2 \ln(U)} \cos (2\pi V), \sqrt{- 2 \ln(U)} \sin(2\pi V)\right)\]
  a la même loi que le couple \((X, Y)\).
\item
  Ecrire une fonction \texttt{box\_muller} permettant de simuler une loi
  \(\mathcal{N}(0, 1)\) en \texttt{R}. Vous comparez l'histogramme
  obtenu à la vrai densité de la loi.
\item
  En déduire, pour tout \(\mu \in \mathbb{R^2}\) et toute matrice
  \(2\times 2\) symmétrique semi-définie positive \(\Sigma\) une méthode
  pour simuler une variable aléatoire
  \(Z\sim \mathcal{N}(\mu, \Sigma)\).
\end{enumerate}

\hypertarget{autour-de-laccepation-rejet}{%
\section{Autour de l'accepation
rejet}\label{autour-de-laccepation-rejet}}

\hypertarget{acceptation-rejet-uxe9tendu-cas-de-deux-fonctions-positives.}{%
\subsection{Acceptation rejet étendu: Cas de deux fonctions
positives.}\label{acceptation-rejet-uxe9tendu-cas-de-deux-fonctions-positives.}}

\emph{Pour cette preuve, vous pourrez mimer les étapes de la preuve dans
le cas usuel, détaillée dans le poly.}

On se propose de montrer que pour que simuler selon une densité par
algorithme d'acceptation rejet, il n'est nécessaire de connaître la
densité qu'\emph{à la constante de normalisation près}. Cette propriété
est très utile dans le cas où le calcul de la constante de normalisation
est coûteux, voir impossible (typiquement en statistiques Bayesiennes).

Plus formellement, soient \(\tilde{f}\) une fonction positive et \(g\)
une densité de probabilité, toutes deux définies sur \(\mathbb{R}^d\)
telles que:

\begin{itemize}
\tightlist
\item
  \(0 < \int_{\mathbb{R}^d}\tilde{f}(x)\text{d}x < \infty\) . On note
  respectivement \(I(\tilde{f})\) cette intégrale et \begin{align*}
  f(x) &= \frac{\tilde{f}(x)}{I(\tilde{f})}
  \end{align*} la densité associée à cette fonction positive.
\item
  Il existe \(M>0\) tel que, pour tout réel \(x\),
  \(\tilde{f}(x) \leq M g(x)\).
\end{itemize}

On note \[\alpha(x) := \frac{\tilde{f}(x)}{Mg(x)}.\] Soit
\((U_m)_{m\geq 1}\) une suite de variables aléatoires i.i.d. de loi
uniforme sur \([0, 1]\). Soit \((Y_m)_{m\geq 1}\) une suite de variables
aléatoires indépendantes et identiquement distributée, de densité donnée
par \(g\). On note \(T\) la variable aléatoire (à valeurs dans
\(\mathbb{N}^*\)):
\[T = \inf\left\lbrace m, \text{ tel que } U_m \leq \alpha(Y_m)\right\rbrace.\].

\begin{enumerate}
\def\labelenumi{\arabic{enumi}.}
\item
  Montrer la variable aléatoire \(X := Y_T\) (\(T\)-ième valeur de la
  suite \((Y_m)_{m\geq 1}\)) a pour densité \(f\).
\item
  Donnez alors la loi de la variable aléatoire \(T\). Quelle est
  l'espérance de \(T\)?
\item
  En déduire un estimateur de \(I(\tilde{f})\) par méthode de Monte
  Carlo, obtenu uniquement à partir de l'algorithme d'acceptation rejet
  défini plus tôt.
\item
  Grâce au théorème central limite, donnez l'expression d'un intervalle
  de confiance asymptotique à 95\% pour \(I(\tilde{f})\), ne dépendant
  d'aucune quantité inconnue.
\end{enumerate}

\hypertarget{recyclage-dans-lacceptation-rejet}{%
\subsection{Recyclage dans l'acceptation
rejet}\label{recyclage-dans-lacceptation-rejet}}

\emph{Dans cette section, on se replace dans le cadre classique de
l'acceptation rejet}.

On se propose d'approcher une intégrale du type:
\(J = \mathbb{E}_f[\varphi(X)]\) où \(f(\) est la densité de la variable
aléatoire \(X\) sur \(\mathbb{R}^d\) selon laquelle on ne sait pas
simuler, et \(\varphi\) est une fonction intégrable par rapport à cette
densité.

À partir d'une densité \(g(x)\) sur \(\mathbb{R}^d\) selon laquelle on
sait simuler, et telle que
\[\exists M>0,\text{ tel que } \forall x\in \mathbb{R}^d,~f(x) \leq Mg(x)\]
on obtient, par algorithme d'acceptation-rejet (pour un tel \(M\) fixé)
un échantillon de variables aléatoires \(i.i.d.\) \(X_1,\dots, X_n\) de
loi donnée par \(f\).

Pour obtenir cet échantillon de taille \(n\), on a simulé \(N\geq n\)
variables aléatoires indépendantes \(Y_1,\dots, Y_N\) de densité \(g\).
On note \(Z_1,\dots,Z_{N - n}\) l'échantillon i.i.d. de variables
aléatoires ayant été rejetées dans l'algorithme d'acceptation rejet.

\begin{enumerate}
\def\labelenumi{\arabic{enumi}.}
\setcounter{enumi}{4}
\item
  Donner l'expression de la densité de la variable aléatoire \(Z_1\).
\item
  En déduire que
  \[\hat{J}_N = \frac{1}{N} \left(\sum_{i = 1}^n \varphi(X_i) + \sum_{j = 1}^{N - n} \frac{(M - 1)f(Z_j)}{Mg(Z_j) - f(Z_j)} \varphi(Z_j) \right)\]
  est un estimateur sans biais de \(J\). Quelle est l'intérêt de cette
  méthode selon vous?
\end{enumerate}

\hypertarget{application}{%
\subsection{Application}\label{application}}

On reprend l'exemple vu en cours d'introduction à la statistique
bayésienne. Vous reprendrez le même modèle ainsi que les mêmes données
utilisées.

\begin{enumerate}
\def\labelenumi{\arabic{enumi}.}
\setcounter{enumi}{6}
\item
  En utilisant le même prior que celui du cours, ainsi que le même loi
  de proposition \(g\), implémentez l'algorithme d'acceptation rejet
  pour tirer selon le posterior. Les algorithmes efficaces seront
  valorisés.
\item
  Implémentez cette méthode, et tracez les densités empiriques des
  posteriors obtenus. Vous donnerez également une estimation de
  \(\mathbb{E}[\theta \vert y_{1:n}]\) ainsi que l'intervalle de
  confiance associé.
\item
  À partir de cette méthode, et en utilisant les questions 3 et 4,
  donner une estimation ainsi qu'un intervalle de confiance pour à 95\%
  de la quantité
  \[\int_{\mathbb{R}^4}\pi(\theta)L(y_{1:n}\theta) \text{d} \theta\]
\item
  Afin d'estimer de \(\mathbb{E}[\theta \vert y_{1:n}]\), implémenter
  l'estimateur \(\hat{J}_N\) de la question 6 (avec le même algorithme
  d'acceptation rejet que pour la question 8). Donnez un intervalle de
  confiance asymptotique pour l'estimation obtenue.
\item
  Comparez les deux estimateurs et commentez.
\end{enumerate}


\end{document}
